\documentclass[letterpaper,12pt]{article}
\usepackage{array}
\usepackage{threeparttable}
\usepackage{geometry}
\geometry{letterpaper,tmargin=1in,bmargin=1in,lmargin=1.25in,rmargin=1.25in}
\usepackage{fancyhdr,lastpage}
\pagestyle{fancy}
\lhead{}
\chead{}
\rhead{}
\lfoot{}
\cfoot{}
\rfoot{\footnotesize\textsl{Page \thepage\ of \pageref{LastPage}}}
\renewcommand\headrulewidth{0pt}
\renewcommand\footrulewidth{0pt}
\usepackage[format=hang,font=normalsize,labelfont=bf]{caption}
\usepackage{listings}
\lstset{frame=single,
  language=Python,
  showstringspaces=false,
  columns=flexible,
  basicstyle={\small\ttfamily},
  numbers=none,
  breaklines=true,
  breakatwhitespace=true
  tabsize=3
}
\usepackage{amsmath}
\usepackage{amssymb}
\usepackage{dsfont}
\usepackage{amsthm}
\usepackage{harvard}
\usepackage{setspace}
\usepackage{float,color}
\usepackage[pdftex]{graphicx}
\usepackage{hyperref}
\hypersetup{colorlinks,linkcolor=red,urlcolor=blue}
\theoremstyle{definition}
\newtheorem{theorem}{Theorem}
\newtheorem{acknowledgement}[theorem]{Acknowledgement}
\newtheorem{algorithm}[theorem]{Algorithm}
\newtheorem{axiom}[theorem]{Axiom}
\newtheorem{case}[theorem]{Case}
\newtheorem{claim}[theorem]{Claim}
\newtheorem{conclusion}[theorem]{Conclusion}
\newtheorem{condition}[theorem]{Condition}
\newtheorem{conjecture}[theorem]{Conjecture}
\newtheorem{corollary}[theorem]{Corollary}
\newtheorem{criterion}[theorem]{Criterion}
\newtheorem{definition}[theorem]{Definition}
\newtheorem{derivation}{Derivation} % Number derivations on their own
\newtheorem{example}[theorem]{Example}
\newtheorem{exercise}[theorem]{Exercise}
\newtheorem{lemma}[theorem]{Lemma}
\newtheorem{notation}[theorem]{Notation}
\newtheorem{problem}[theorem]{Problem}
\newtheorem{proposition}{Proposition} % Number propositions on their own
\newtheorem{remark}[theorem]{Remark}
\newtheorem{solution}[theorem]{Solution}
\newtheorem{summary}[theorem]{Summary}
%\numberwithin{equation}{section}
\bibliographystyle{aer}
\newcommand\ve{\varepsilon}
\newcommand\boldline{\arrayrulewidth{1pt}\hline}


\begin{document}

\begin{flushleft}
  \textbf{\large{Problem Set \#1}} \\
  DSGE Models, Prof. Kerk Phillips \\
  Martina Fraschini
\end{flushleft}

\vspace{5mm}

\noindent\textbf{Exercise 1}\\
In the Brock and Mirman's model the households solves the following dynamic program:
\[ V\left(K_{t}, z_{t}\right)=\max _{K_{t+1}} \ln \left(e^{z_{t}} K_{t}^{\alpha}-K_{t+1}\right)+\beta E_{t}\left\{V\left(K_{t+1}, z_{t+1}\right)\right\}, \]
where the law of motion is:
\[ z_{t+1}=\rho z_{t}+\varepsilon_{t} ; \quad \varepsilon_{t} \sim i . i . d\left(0, \sigma^{2}\right). \]
The associated Euler equation is:
\[ \frac{1}{e^{z_{t}} K_{t}^{\alpha}-K_{t+1}}=\beta E_{t}\left\{\frac{\alpha e^{z_{t+1}} K_{t+1}^{\alpha-1}}{e^{z_{t+1}} K_{t+1}^{\alpha}-K_{t+2}}\right\}. \]
In order to find an algebraic solution for the policy function we use the ``guess and verify" method. We guess that the policy function is in the form of $K_{t+1}=A e^{z_{t}} K_{t}^{\alpha}$ and we substitute it in the Euler equation. We obtain:
\[ \frac{1}{e^{z_{t}} K_{t}^{\alpha}-A e^{z_{t}} K_{t}^{\alpha}}=\beta E_{t}\left\{\frac{\alpha e^{z_{t+1}} A^{\alpha-1} e^{(\alpha-1)z_{t}} K_{t}^{\alpha^2-\alpha}}{e^{z_{t+1}} A^\alpha e^{\alpha z_{t}} K_{t}^{\alpha^2}-A^{1+\alpha} e^{z_{t+1}}e^{\alpha z_{t}} K_{t}^{\alpha^2}}\right\}, \]
\[ \frac{1}{(1-A) e^{z_{t}} K_{t}^{\alpha}}=\beta E_{t}\left\{\frac{\alpha A^{-1}e^{-z_t} K_t^{-\alpha} A^\alpha e^{\alpha z_{t}} K_{t}^\alpha e^{z_{t+1}}}{(1-A)A^\alpha e^{\alpha z_{t}} K_{t}^{\alpha^2}e^{z_{t+1}}}\right\}, \]
\[ \frac{1}{(1-A) e^{z_{t}} K_{t}^{\alpha}}=\beta E_{t}\left\{\frac{\alpha}{(1-A)A e^{z_{t}} K_{t}^{\alpha}}\right\}, \]
\[ \alpha\beta=\frac{(1-A)A e^{z_{t}} K_{t}^{\alpha}}{(1-A) e^{z_{t}} K_{t}^{\alpha}}, \]
\[ A=\alpha\beta. \]
Therefore the policy function is $K_{t+1}=\alpha\beta~ e^{z_{t}} K_{t}^{\alpha}$.\\

\noindent\textbf{Exercise 2}\\
Our baseline model with:
\[ u\left(c_{t}, \ell_{t}\right)=\ln c_{t}+a \ln \left(1-\ell_{t}\right), \]
and \[ f\left(K_{t}, L_{t}, z_{t}\right)=e^{z_{t}} K_{t}^{\alpha} L_{t}^{1-\alpha}, \]
has the following characterizing equations:
\[ c_{t}=(1-\tau)\left[w_{t} \ell_{t}+\left(r_{t}-\delta\right) k_{t}\right]+k_{t}+T_{t}-k_{t+1}, \]
\[ \frac{1}{c_t}=\beta E_{t}\left\{\frac{1}{c_{t+1}}\left[\left(r_{t+1}-\delta\right)(1-\tau)+1\right]\right\}, \]
\[ -\frac{a}{1-\ell_{t}}=\frac{1}{c_t} w_{t}(1-\tau), \]
\[ r_{t}=\alpha e^{z_{t}} k_{t}^{\alpha-1} \ell_{t}^{1-\alpha}, \]
\[ w_{t}=(1-\alpha)e^{z_{t}} k_{t}^{\alpha} \ell_{t}^{-\alpha}, \]
\[ \tau\left[w_{t} \ell_{t}+\left(r_{t}-\delta\right) k_{t}\right]=T_{t}, \]
\[ z_{t}=\left(1-\rho_{z}\right) \overline{z}+\rho_{z} z_{t-1}+\epsilon_{t}^{z} ; \quad \epsilon_{t}^{z} \sim \text { i.i.d.}\left(0, \sigma_{z}^{2}\right). \]
In this case we can't use the same tricks as in Exercise 1 to solve for the policy function because the model is too complex and it's impossible to guess a good algebraic solution.\\

\noindent\textbf{Exercise 3}\\
Our baseline model with:
\[ u\left(c_{t}, \ell_{t}\right)=\frac{c_{t}^{1-\gamma}-1}{1-\gamma}+a \ln \left(1-\ell_{t}\right), \]
and \[ f\left(K_{t}, L_{t}, z_{t}\right)=e^{z_{t}} K_{t}^{\alpha} L_{t}^{1-\alpha}, \]
has the following characterizing equations:
\[ c_{t}=(1-\tau)\left[w_{t} \ell_{t}+\left(r_{t}-\delta\right) k_{t}\right]+k_{t}+T_{t}-k_{t+1}, \]
\[ c_t^{-\gamma}=\beta E_{t}\left\{c_{t+1}^{-\gamma}\left[\left(r_{t+1}-\delta\right)(1-\tau)+1\right]\right\}, \]
\[ -\frac{a}{1-\ell_{t}}=c_t^{-\gamma} w_{t}(1-\tau), \]
\[ r_{t}=\alpha e^{z_{t}} k_{t}^{\alpha-1} \ell_{t}^{1-\alpha}, \]
\[ w_{t}=(1-\alpha)e^{z_{t}} k_{t}^{\alpha} \ell_{t}^{-\alpha}, \]
\[ \tau\left[w_{t} \ell_{t}+\left(r_{t}-\delta\right) k_{t}\right]=T_{t}, \]
\[ z_{t}=\left(1-\rho_{z}\right) \overline{z}+\rho_{z} z_{t-1}+\epsilon_{t}^{z} ; \quad \epsilon_{t}^{z} \sim \text { i.i.d.}\left(0, \sigma_{z}^{2}\right). \]\\

\noindent\textbf{Exercise 4}\\
Our baseline model with:
\[ u\left(c_{t}, \ell_{t}\right)=\frac{c_{t}^{1-\gamma}-1}{1-\gamma}+a \frac{\left(1-\ell_{t}\right)^{1-\xi}-1}{1-\xi}, \]
and \[ f\left(K_{t}, L_{t}, z_{t}\right)=e^{z_{t}}\left[\alpha K_{t}^{\eta}+(1-\alpha) L_{t}^{\eta}\right]^{\frac{1}{\eta}}, \]
has the following characterizing equations:
\[ c_{t}=(1-\tau)\left[w_{t} \ell_{t}+\left(r_{t}-\delta\right) k_{t}\right]+k_{t}+T_{t}-k_{t+1}, \]
\[ c_t^{-\gamma}=\beta E_{t}\left\{c_{t+1}^{-\gamma}\left[\left(r_{t+1}-\delta\right)(1-\tau)+1\right]\right\}, \]
\[ -a(1-\ell_{t})^{-\xi}=c_t^{-\gamma} w_{t}(1-\tau), \]
\[ r_{t}=\alpha e^{z_{t}}, \]
\[ w_{t}=(1-\alpha)e^{z_{t}}, \]
\[ \tau\left[w_{t} \ell_{t}+\left(r_{t}-\delta\right) k_{t}\right]=T_{t}, \]
\[ z_{t}=\left(1-\rho_{z}\right) \overline{z}+\rho_{z} z_{t-1}+\epsilon_{t}^{z} ; \quad \epsilon_{t}^{z} \sim \text { i.i.d.}\left(0, \sigma_{z}^{2}\right). \]\\

\noindent\textbf{Exercise 5}\\
Assuming $\ell_t=1$ and considering the market clearing condition $L_t=\ell_t=1$, we have that our baseline model with:
\[ u\left(c_{t}\right)=\frac{c_{t}^{1-\gamma}-1}{1-\gamma}, \]
and \[ f\left(K_{t}, L_{t}, z_{t}\right)=K_{t}^{\alpha}\left(e^{z_{t}}\right)^{1-\alpha}, \]
has the following characterizing equations:
\[ c_{t}=(1-\tau)\left[w_{t} +\left(r_{t}-\delta\right) k_{t}\right]+k_{t}+T_{t}-k_{t+1}, \]
\[ c_t^{-\gamma}=\beta E_{t}\left\{c_{t+1}^{-\gamma}\left[\left(r_{t+1}-\delta\right)(1-\tau)+1\right]\right\}, \]
\[ c_t^{-\gamma} w_{t}(1-\tau)=0, \]
\[ r_{t}=\alpha k_{t}^{\alpha-1}\left(e^{z_{t}}\right)^{1-\alpha}, \]
\[ w_{t}=0, \]
\[ \tau\left[w_{t}+\left(r_{t}-\delta\right) k_{t}\right]=T_{t}, \]
\[ z_{t}=\left(1-\rho_{z}\right) \overline{z}+\rho_{z} z_{t-1}+\epsilon_{t}^{z} ; \quad \epsilon_{t}^{z} \sim \text { i.i.d.}\left(0, \sigma_{z}^{2}\right). \]
The steady state version of this equations is:
\[ \bar{c}=(1-\tau)\left[\bar{w} +\left(\bar{r}-\delta\right) \bar{k}\right]+\bar{T}, \]
\[ \bar{c}^{-\gamma}=\beta E_{t}\left\{\bar{c}^{-\gamma}\left[\left(\bar{r}-\delta\right)(1-\tau)+1\right]\right\}, \]
\[ \bar{c}^{-\gamma} \bar{w}(1-\tau)=0, \]
\[ \bar{r}=\alpha \bar{k}^{\alpha-1}\left(e^{\bar{z}}\right)^{1-\alpha}, \]
\[ \bar{w}=0, \]
\[ \tau\left[\bar{w}+\left(\bar{r}-\delta\right) \bar{k}\right]=\bar{T}. \]

\end{document}

